%--------------------------------------------------------------------
% Vorlage für eine Abschlussarbeit
% Darf ohne Einschränkungen weiterverwendet und abgeändert werden.
% mehr zur Verwedung in der README.md
% ⓒ (CC0) 2016, 2018 Henning Iseke <h_i_@online.de>
%--------------------------------------------------------------------

\documentclass[11pt, % 11 Punkt Schriftgröße
german, % standardmäßig deutsche Eigenarten einhalten
abstract=true, % Abstract verwenden
twoside, % Zweiseitig
captions=tableheading, % Tabellenüberschriften mit richtigem Abstand
BCOR=10mm, % TODO: Bindekorrektur, muss je nach Bindung angepasst werden
%draft % TODO: Entwurfsmodus am Ende entfernen
]{scrreprt}

\usepackage{ifluatex} % Paket, um Vorlage für Lua- und pdfLaTeX zu verwenden
\ifluatex % LuaTeX
    %\usepackage{fontspec} % Schriftarten verwalten (es können im Grunde genommen alle installierten Schriftarten verwendet werden)
    %\setmainfont[Ligatures=TeX]{Linux Libertine O} % Hauptschriftart setzen
    %\setsansfont[Ligatures=TeX]{Linux Biolinum O} % Serifenlose Schriftart

    \usepackage{polyglossia} % Namen, Silbentrennung an Sprache anpasen
    \setdefaultlanguage[spelling=new,babelshorthands=true]{german} % Deutsch als Standardsprache
    \setotherlanguage{english} % weitere Sprache (für Abstract)

    \newcommand*{\compiler}{Lua\LaTeX}
\else % kein LuaTeX
    \usepackage[utf8]{inputenc} % Kodierung der Datei
    \usepackage[T1]{fontenc} % Vollen Umfang der Schriftzeichen
    \usepackage[main=ngerman,english]{babel} % Sprache auf Deutsch (neue Rechtschreibung)
    \newenvironment{english}{\begin{otherlanguage}{english}}{\end{otherlanguage}}
    \newcommand*{\compiler}{pdf\LaTeX}
    %\usepackage{libertine} % Schriftart Linux Libertine/Biolinum verwenden
\fi

\usepackage{siunitx} % Darstellung von Größen
\sisetup{locale=DE, % in Deutschland übliche Darstellung
separate-uncertainty, % Unsicherheit mit ± getrennt darstellen
table-number-alignment=center, % Zahlen in Tabellen zentrieren (mit Option S)
}

\usepackage{graphicx} % Einbinden von Grafiken
\usepackage{pdfpages} % Einbinden von PDF-Grafiken

\usepackage{booktabs} % schöne Tabellen mit \toprule, \midrule, \bottomrule
\usepackage{multirow}
\usepackage{multirow} % mehrzeilige Zellen
\usepackage{csquotes} % Anführungszeichen mit \enquote
\usepackage[automark]{scrlayer-scrpage} % Seiten mit Kopf- und Fußzeile gestalten
\pagestyle{scrheadings} % Seitenstil setzen
%\usepackage{microtype} % Mikrotypographie

% Literaturangaben
\usepackage[backend=biber, % Biber zum Erstellen verwenden: biber <Dokument>
style=phys, % in der Physik üblichen Stil
articletitle=false, % ohne Titel bei Artikeln (bei Büchern, … schon)
pageranges=false, % nur Anfangsseitenzahl
language=german, % deutsche Sprache
biblabel=brackets % Verweise als „[1]“
]{biblatex} % Literaturverzeichnis
\addbibresource{literatur} % bib-Datei mit den Infos zur Literatur
%\renewcommand*{\bibfont}{\small} % kleine Schriftart für Literaturverzeichnis

% Mathematik
\usepackage{amsmath} % Umgebungen/Befehle, die für Mathe nützlich sind
\usepackage{amssymb}
\ifluatex
\usepackage{unicode-math} % Symbole für Mathematik
%\setmathfont{Asana-Math.otf} % Schriftart für den Mathemodus (nur mit LuaLaTeX)
\fi
\usepackage{physics} % Umgebungen/Befehle für Physik
\usepackage{cancel}

\usepackage{caption} % Anpassung der Bildunterschriften, Tabellenüberschriften
\usepackage{subcaption} % mehrere Unterabbilungen in einer Abbildung, auch Tabellen

\usepackage[hidelinks,pdfauthor={\autor},
pdfsubject=Abschlussarbeit,pdftitle={\titel},pdfkeywords={\schlagworte}
unicode=true]{hyperref} % Metadaten und Links in PDFs

% Nummerierung anpassen
\numberwithin{equation}{chapter}
\numberwithin{figure}{chapter}
\numberwithin{table}{chapter}
